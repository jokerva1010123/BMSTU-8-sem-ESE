\section*{Содержание проекта}
Команда разработчиков из 16 человек занимается созданием карты города на основе собственного модуля отображения. Проект должен быть завершен в течение 6 месяцев. Бюджет проекта: 50 000 рублей.

\section*{Сведения о проекте}

\img{0.5}{1}{Краткая информация о ресурсах проекта}

Ресурсы «Системный аналитик», «Художник-дизайнер», «Технический писатель» были одновременно задействованы в разных задачах, из-за чего возникли перегрузки.

\img{0.45}{2}{Наложение задач}

Устранить перегрузки можно следующими способами:
\begin{itemize}[label = ---]
    \item изменить календарь работы ресурса;
    \item назначить ресурс на неполный рабочий день;
    \item изменить профиль назначения ресурса;
    \item изменить ставку оплаты ресурса;
    \item добавить ресурсу время задержки;
    \item разбить задачу на этапы и перекрыть по времени их выполнение;
    \item применить автоматическое выравнивание.
\end{itemize}

\section*{Задание 1. Выравнивание загрузки ресурсов в проект}

После возникшей перегрузки ресурсов необходимо выполнить их выравнивание. Для выравнивания ресурсов будет использоваться автоматическое выравнивание, которое работает по принципу неизменения критического пути.

\img{0.9}{3}{Параметры выравнивания}
После автоматического выравнивания дата завершения проекта сдвинулась c 19.09.24 на 23.09.24, бюджет проекта не изменился, что можно наблюдать на рисунке 4.

\img{0.475}{4}{Состояние проекта после выравнивания}

На рисунке 5 зеленым цветом показаны задачи до выравнивания, голубым – после.

\img{0.5}{5}{Результат автоматического выравнивания}

На рисунке 6 видно, что работники больше не перегружены.

\img{0.45}{6}{Анализатор ресурсов}
\newpage
\section*{Задание 2. Учет периодических задач в плане проекта}

Добавлена еженедельная задача \textbf{Совещание}, длительностью 1 час по средам.

\img{0.8}{7}{Создание повторяющейся задачи}

\img{0.4}{8}{Повторяющаяся задача в списке задач}
\newpage
Были назначены ресурсы всем сотрудникам, кроме наборщиков данных и программистов №1-4.

\img{0.8}{9}{Назначение ресурсов}

\img{0.6}{10}{Затраты на совещания}

Бюджет проекта увеличился до 68 217р, что превышает заявленные 50 000р.

\img{0.5}{11}{Состояние проекта после добавления совещаний}

Также произошли перегрузки сотрудников в связи с тем, что совещания происходили во время их работы.

\img{0.5}{12}{Перегрузки сотрудников после добавления совещаний}
\newpage
Чтобы устранить перегрузки, было выполнено выравнивание ещё раз. Поиск превышений доступности был выставлен «по неделям».

\img{0.35}{13}{Результат выравнивания}

Для всех участников совещания был создан план затрат В, в котором нет затрат на использование. Данный план затрат был применен к ресурсам совещания.

\img{0.8}{14}{Настройка другого плана оплаты}

\img{0.9}{15}{План затрат на совещания}
\newpage
После этого затраты на совещания снизились с 20 039 до 1 769 рублей.

\img{0.6}{16}{Обновленный план затрат на совещания}

Бюджет проекта уменьшился до 49 667 р, что укладывается в заявленные 50 000 р.

\img{0.6}{17}{Состояние проекта}

\section*{Задание 3. Оптимизация критического пути}

На критическом пути одними из самых длинных являются задачи, связанные с программированием. При уменьшение их длительности уменьшилась бы и длительность всего проекта.

\img{0.7}{18}{Критический путь}

В анализаторе ресурсов можно заметить, что программисты используются нерационально.

\img{0.33}{19}{Анализатор ресурсов}
\newpage
Назначим программистов на задачи критического пути, в которых задействованы не все программисты (7, 14, 16 и 26).

\img{0.65}{20}{Назначение ресурсов задаче «Программирование интерфейса»}

Полученная занятость работников представлена на рисунке 21. Теперь программисты заняты более равномерно и большую часть времени работают все четверо.

\img{0.4}{21}{Анализатор ресурсов}

Таккже после оптимизаций удалим совещания, которые происходят после завершения последней задачи.

\img{0.6}{22}{Оставшиеся совещания в проекте}

После добавления дополнительных программистов к задачам, дата сдвинулась на 29.07.2024, а затраты стали равными 48 564 р. Теперь проект удовлетворяет требованииям по срокам реализации и бюджету.

\img{0.6}{23}{Состояние проекта}

На рисунках 24–27 представлены диаграммы затрат и трудозатрат до и после оптимизации критического пути.

По преведенным диаграммам проведем анализ и сравним с предыдущими результатами. Удалось сократить на 1\% затраты группы «Программирование», при этом на 1\% увеличились затраты групп «Анализ данных» и «Дизайн». Также удалось сократить на 1\% трудозатраты группы «М-медиа», при этом на 1\% увеличились трудозатраты группы «Ввод данных».
\newpage
\img{0.8}{24}{Новая диаграмма затрат}
\img{0.7}{25}{Старая диаграмма затрат}
\newpage
\img{0.8}{26}{Новая диаграмма трудозатрат}
\img{0.7}{27}{Старая диаграмма трудозатрат}
\newpage
Сохраняем базовый план проекта. При сохранении базового плана сохраняется полный набор предварительных оценок проекта, которые в дальнейшем будут использоваться для контроля за изменениями.
\img{1}{28}{Старая диаграмма трудозатрат}

\section*{Выводы}

Было обнаружено, что программисты используют половину бюджета. Следовательно, для сокращения бюджета стоит сократить время на задачи, исполняемые программистами путем увеличения трудозатрат на задачу. Также при таком подходе сокращается время проекта, так как задачи выполняются быстрее.

При добавлении совещений требуется предусмотреть создание отдельного плана оплат, чтобы не оплачивать работникам часы совещаний как полноценные часы работы.

В результате оптимизации критического пути удалось снизить затраты на группу «Программисты» на 1\%, что существенно, так как затраты на них занимали половину от общего бюджета проекта.

Проведя все оптимизации, удалось сократить бюджет проекта с 67 987 до 48 564 рублей и время проекта с 29.7 до 20.77 недель (Проект завершается 29.07.2024).