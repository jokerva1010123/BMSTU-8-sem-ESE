\chapter{Исследовательская часть}
В данном разделе будут приведены примеры работы программа, а также проведены замеры процессорного времени и количества сравнений.

\section{Технические характеристики}

Технические характеристики устройства, на котором выполнялось тестирование:

\begin{itemize}
	\item[---] операционная система Window 10 Home Single Language;
	\item[---] память 8 Гб;
	\item[---] процессор 11th Gen Intel(R) Core(TM) i5-1135G7 2.42 ГГц, 4 ядра.
\end{itemize}

Во время замера устройство было подключено к сети электропитания, нагружено приложениями окружения и самой системой замера.


\section{Демонстрация работы программы}

На рисунках \ref{img:p1} и \ref{img:p2} представлены результаты работы программы.

\img{50mm}{p1}{Пример 1 работы программы}
\img{50mm}{p2}{Пример 2 работы программы}

\section{Время выполнения реализаций алгоритмов поиска ключа}
Как было сказано выше, для замера времени выполнения части кода используется функция process\_time() из библиотеки time \cite{pythonlangtime}. 

На рисунке \ref{img:graph} представлен график зависимости времени поиска от
индекса ключа словаря, для построения которого использовались данные
о времени поиска каждого элемента в словаре. Индекс ключа
указан на горизонтальной оси.

\img{80mm}{graph}{Время работы алгоритмов для поиска каждого ключа словаря}

\newpage

\section{Количество сравнений}
В ходе эксперимента было подсчитано количество сравнений, которые понадобились, чтобы найти каждый ключ в словаре, и на основе полученных данных составлены гистограммы.

Гистограмма для алгоритма поиска в словаре полным перебором представлены на рисунке \ref{img:png1}.

\img{170mm}{png1}{Количество сравнений поиска с помощью полного перебора}


Гистограммы для алгоритма бинарного поиска в словаре представлены на рисунках \ref{img:png2} и \ref{img:png3}.

\img{180mm}{png2}{Количество сравнений бинарного поиска, сортировка по расположению}
\img{180mm}{png3}{Количество сравнений бинарного поиска, сортировка по количеству сравнений}

\clearpage

\section{Вывод}
В результате эксперимента и полученных графиков и гистограмм, приведенных выше, видно, что реализация алгоритма полного перебора медленнее, чем реализация алгоритма бинарного поиска. Время растет линейно и увеличивается с увеличением индекса элемента словаря. Также растет количество сравнений, от которого напрямую зависит времени работы программы.

Алгоритм бинарного поиска требует дополнительных расходов времени на подготовку данных к работе с алгоритмом, но эти расходы можно не учитывать на этапе поиска (это предварительная подготовка данных).