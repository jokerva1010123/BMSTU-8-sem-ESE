\chapter{Тренировочное задание (вариант 0)}
Осуществить планирование проекта со следующими временными характеристиками:
\img{0.8}{task}{Временные характеристики}

Дата начала проекта – 1-й рабочий день марта текущего года. Провести планирование работ проекта, учитывая следующие связи между задачами:
\begin{enumerate}
    \item Предусмотреть, что A, E и F – исходные работы проекта, которые можно начинать одновременно.
    \item Работы B и I начинаются сразу по окончании работы F.
    \item Работа J следует за E, а работа C – за A.
    \item Работы H и D следуют за B, но не могут начаться, пока не завершена C.
    \item Работа G начинается после завершения H и J.
\end{enumerate}
По умолчанию используется фиксированный объем ресурсов.
\img{0.4}{test}{Решение тренировочного задания}
\chapter{Задание лабораторной работы}
\section*{Содержание проекта}
Команда разработчиков из 16 человек занимается созданием карты города на основе собственного модуля отображения. Проект должен быть завершен в течение 6 месяцев. Бюджет проекта: 50 000 рублей.
\section*{Задание 1. Настройка рабочей среды проекта}
На вкладке \textbf{Проект} -> \textbf{Сведения} внесены параметры по условию.
\img{0.8}{1}{Настройка сведений о проекте}

На вкладке \textbf{Файл} -> \textbf{Параметры} -> \textbf{Расписание} установлены параметры рабочей недели и планирования.
\img{0.7}{2}{Настройка расписания}

На вкладке \textbf{Проект} -> \textbf{Изменить рабочее время} установлены нерабочие праздничные дни.
\img{0.7}{3}{Настройка нерабочих праздничных дней}

На вкладке \textbf{Задача} -> \textbf{Суммарная задача} установлена суммарная задача проекта и добавлена заметка с основной информацией о проекте.
\img{0.7}{4}{Настройка суммарной задачи}

\section*{Задание 2. Создание списка задач}
Введен список задач в соответствии с таблицей, представленной в задании лабораторной работы.
\img{0.45}{5}{Список задач}

\section*{Задание 3. Структурирование списка задач}
При помощи кнопки \textbf{Понизить уровень задачи} были выделены подзадачи в соответствии с условием.
\img{0.45}{6}{Разбиение на подзадачи}

\section*{Задание 4. Установление связей между задачами}
При помощи заполнения колонки \textbf{Предшественник} у каждой задачи были установлены связи между задачами.\
\img{0.35}{7}{Установленные связи между задачами}

\section*{Выводы}
В данной лабораторной работе были освоены возможности программы \textbf{Microsoft Project} для планирования проекта по разработке программного обеспечения. Был создан план проекта создания карты города. Была получена дата завершения работ – 16.09.24. В итоге длительность проекта составила 6 месяцев и 16 дней.
