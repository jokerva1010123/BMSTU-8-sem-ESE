\chapter*{Введение}
\addcontentsline{toc}{chapter}{Введение}

Словарь, как тип данных, применяется везде, где есть связь “ключ -- значение” или “объект -- данные”: поиск налогов по ИНН и другое. Поиск --- основная задача при использовании словаря. Данная задача решается различными способами, которые дают различную скорость решения.

Со временем стали разрабатывать алгоритмы поиска в словаре. В данной лабораторной работе мы рассмотрим два алгоритма.
\begin{enumerate}
	\item Поиск полным перебором.
	\item Бинарный поиск.
\end{enumerate}

Цель данной работы --- получить навык работы со словарём, как структурой данных, реализовать алгоритмы поиска по словарю (указаны выше) с применением оптимизаций.

Для достижения цели поставлены следующие задачи.
\begin{enumerate}
	\item Изучить два алгоритма поиска в словаре.
	\item Привести схемы алгоритмов поиска в словаре.
	\item Описать структуру разрабатываемого программного обеспечения.
	\item Применить изученные основы для реализации поиска значений в словаре по ключу.
	\item Провести функциональное тестирование реализации разработанного алгоритма.
	\item Получить замеры количества сравнений для каждого ключа (для всех двух алгоритмов).
	\item Провести сравнительный анализ по времени для реализованных алгоритмов.
	\item Подготовить отчет по лабораторной работе.
\end{enumerate}