\chapter{Тренировочное задание (вариант 0)}
\begin{enumerate}
    \item Дополнить временной план проекта, подготовленный на предыдущем этапе (лабораторная работа №1), информацией о ресурсах и определить стоимость проекта.
    \item Для этого заполнить ресурсный лист в программе MS Project, принимая во внимание, что к реализации проекта привлекается не более 10 человек.
    \item Предусмотреть, что стандартная ставка ресурса составляет 150руб./д
    \item Произвести назначение ресурсов на задачи в соответствии с таблицей. С учетом того, что квалификация ресурсов одинаковая, при назначении ресурсов использовать процент загрузки.
    \img{0.6}{1}{Количество исполнителей для задач}
    \item Запланировать для выполнения работы Е использование материального ресурса, стоимостью 300 руб. за единицу и с нормой расхода 4 единицы в день.
\end{enumerate}

Был заполнен ресурсный лист (добавлены трудовые и материальные ресурсы).
\img{0.55}{2}{Заполнение ресурсного листа}
\newpage
Было произведено назначение ресурсов.
\img{1.1}{3}{Пример назначения ресурсов}
\img{0.65}{4}{Задачи проекта}
\img{0.6}{5}{Диаграмма Ганта}
\newpage
При выполнении задания возникли перегрузки. Это связано с нехваткой ресурсов при одновременном выполнении работ. В визуальном оптимизаторе ресурсов можно увидеть наложение задач исполнителей.
\img{0.4}{6}{Наложение задач}

\chapter{Задание лабораторной работы}
\section*{Задание 1. Создание списка ресурсов}
В соответствии с заданием был заполнен ресурсный лист проекта.
\img{0.5}{7}{Список ресурсов}

\section*{Задание 2. Назначение ресурсов задачам}
Все ресурсы были назначены задачам в соответствии с таблицей.
\img{0.5}{8}{Диаграмма Ганта}

Ресурсы «Системный аналитик», «Художник-дизайнер», «Технический писатель» были одновременно задействованы в разных задачах, из-за чего возникли перегрузки.
\img{0.45}{9}{Наложение задач}

Задачам 2, 8 и 12 было задано по 1000 р. фиксированных затрат.

Был добавлен новый трудовой ресурс «Аренда сервера». Стоимость аренды – 2 рубля в час.
\img{0.55}{10}{Добавление нового ресурса}

Для задачи №8 «Построение базы объектов» был арендован дополнительный сервер.
\img{1.0}{11}{Назначение нового ресурса}
\img{0.45}{12}{Состояние проекта после назначения ресурсов}
\newpage

\section*{Задание 3. Анализ затрат по группам ресурсов}
Была проведена структуризация затрат по группам ресурсов.
\img{0.45}{12}{Лист ресурсов}

После группировки данных по группам ресурсов были получены диаграммы информации о затратах и трудозатратах.
\img{0.7}{14}{Информация о затратах}
\img{0.7}{15}{Информация о трудозатратах}

\section*{Выводы}
На аренду сервера уходит значительная часть бюджета (14\%). Группы «Программирование» и «Ввод данных» имеют схожие трудозатраты (29\% и 25\% соответственно). При этом затраты на «Программированиие» составили половину от всего бюджета проекта, в то время как затраты на «Ввод данных» составили всего 11\% бюджета. На груупу «Анализ» ушло сопоставимое кол-во затрат (10\%), однакоко трудозатраты составили всего 2\%, что в 12.5 раз меньше трудозатрат на «Ввод данных».

При помощи программных средств была выявлена перегрузка определённых работников, т. к. они выполняют несколько задач одновременно. Требуется оптимизировать процесс планирования и распределения задач. 

Затраты проекта составили 48 286 рубля, значит он уложился в бюджет, который составлял 50 000 рублей. Трудозатраты составили 9 473 ч.
