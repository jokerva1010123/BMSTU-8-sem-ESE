\chapter{Задания}

\section*{Задание преподавателя}

Дата отчета \textbf{12 мая}.

Отметить как выполненные все работы, которые должны были завершиться на эту дату, кроме:
\begin{enumerate}
	\item 6 задача завершилась 11 апреля.
	\item 3 апреля для задачи «Создание мультимедиа наполнения» купили специализированное ПО 
	стоимостью 600 рублей и еще 100 рублей понадобилось на его установку.
	\item С 10 апреля на 10\% была увеличена зарплата мультимедиа-корреспондента.
	\item С 3 апреля заказчик заменил совещания на презентации. Они проходят раз в 2 недели 
	по понедельникам в течение 2 часов. На них присутствуют: ведущий программист, аналитик, 
	мультимедиа-корреспондент и веб-дизайнер. Для подготовки раздаточных материалов для 
	участников презентаций необходима бумага в объеме 2 пачки на одну презентацию 
	(стоимость - 100 руб. за пачку).
	\item С 13 марта на 5\% увеличилась стоимость аренды сервера.
\end{enumerate}


\section{Актуализация параметров проекта}

Задана дата отчета (рисунок~\ref{img:task4_1_report_date}).
\imgw{task4_1_report_date}{ht!}{0.7\textwidth}{Дата отчета}

6 задаче установлена фактическая дата завершения (рисунок~\ref{img:task4_1_task6_res}).
\imgw{task4_1_task6_res}{ht!}{0.7\textwidth}{Установка фактической даты завершения 6 задачи}

\newpage
Добавлен ресурс <<Специализированное ПО>> (рисунки~\ref{img:task4_1_2_po_add}-\ref{img:task4_1_2_task17_cost}).
\imgw{task4_1_2_po_add}{ht!}{0.7\textwidth}{Материальный ресурс <<Специализированное ПО>>}
\imgw{task4_1_2_task17_cost}{ht!}{0.7\textwidth}{Ресурсы задачи «Создание мультимедиа наполнения»}

С 10 апреля на 10\% была увеличена зарплата мультимедиа-корреспондента (рисунок~\ref{img:task4_1_3_salary}).
\imgw{task4_1_3_salary}{ht!}{0.8\textwidth}{Зарплата мультимедиа-корреспондента}

Замена совещаний на презентации с 3 апреля (рисунок~\ref{img:task4_4_ret_task_resourses}).
\imgw{task4_4_ret_task_resourses}{ht!}{0.8\textwidth}{Замена совещаний на презентации}

Было произведено автоматическое выравниевание ресурсов, перегрузки устранены.
Ресурсам презентаций также был установлен план затрат B,
стоимость презентаций уменьшилась (рисунок~\ref{img:task4_4_ret_task_after_align}).
\imgw{task4_4_ret_task_after_align}{ht!}{0.8\textwidth}{Стоимость презентаций с планом затрат B}

С 13 марта на 5\% увеличилась стоимость аренды сервера (рисунок~\ref{img:task4_5_salary}), 
а общая сумма проекта выросла до 50 805,88 рублей (рисунок~\ref{img:task4_5_gen_cost})

\imgw{task4_5_salary}{ht!}{0.8\textwidth}{Стоимость аренды серевера}
\imgw{task4_5_gen_cost}{ht!}{0.8\textwidth}{Стоимость проекта}

По результатам актуализации параметров проекта срок проекта не был превышен, а стоимость проекта превысила запланированный бюджет.
Для оптимизации затрат проекта были сокращены затраты на презентации посредством исключения из презентаций сотрудников, которые уже 
окончили свои работы на момент проведения презентации: системного аналитика --- с 17.04.23, мультимедиа-корреспондента --- с 13.06.23 
(рисунок~\ref{img:task4_6_ret_cost_1}).
\imgw{task4_6_ret_cost_1}{ht!}{0.8\textwidth}{Стоимость презентаций после исключения из последующих презентаций сотрудников, окончивших все свои работы}

Далее были добавлены 2 программиста (стало 6) на задачи,  связанные с программированием, для сокращения трудозатрати затрат соответвенно,
т.к. они  являются высокооплачиваемыми специалистами. Это помогло ускорить завершение проекта, но не сильно уменьшило стоимость проекта.
Т.к. сервер арендуется на время выполнения <<Построения базы объектов>>, а наборщики выполняют последнюю из ее подзадач, увеличение числа
наборщиков данных на 2 (стало 7) позволило приблизить дату окончания проекта. Т.к. наборщики являются низкооплачиваемыми специалистами, а 
аренда сервера на время их работы стоит дорого, сократив время работы наборщиков путем их увеличения, стоимость проекта уменьшилась и больше 
не превышает запланированный бюджет (рисунок~\ref{img:task4_7_gen_cost}). Распредение задач между программистами и наборщиками показаны 
на рисунках \ref{img:task4_7_view_1}-\ref{img:task4_7_view_2}.

\imgw{task4_7_gen_cost}{ht!}{0.7\textwidth}{Стоимость проекта}

\imgw{task4_7_view_1}{ht!}{0.7\textwidth}{Визуальный оптимизатор (часть 1)}
\imgw{task4_7_view_2}{ht!}{0.7\textwidth}{Визуальный оптимизатор (часть 2)}


Проект обновлен (для задач отмечен процент завершения --- рисунок~\ref{img:task4_9_upd_poject}).
\imgw{task4_9_upd_poject}{ht!}{1\textwidth}{Настройки обновления проекта}

На экран выведена линия прогресса (рисунок~\ref{img:task4_9_lines}). По ней видно, что проект не отклонился от графика.
\imgw{task4_9_lines}{ht!}{1\textwidth}{Линия прогресса}

Таким образом, стоимость проекта составила \textbf{49 132,40 рубля} (укладывается в бюджет), а дата окончания проекта --- \textbf{05.07.23} (сдвинулась на 15 дней, 
укладывается в срок).




% Информация о трудозатратах по группам ресурсов представлена в 
% графическом виде для проекта на моменты его конечных состояний из ЛР №2 и ЛР №3 (рисунки~\ref{img:task2_3_group_laborcost}-\ref{img:task3_3_group_laborcost}).

% \imgw{task2_3_group_laborcost}{ht!}{0.7\textwidth}{Информация о трудозатратах (ЛР №2)}
% \imgw{task3_3_group_laborcost}{ht!}{0.7\textwidth}{Информация о трудозатратах (ЛР №3)}

% \newpage
% По результатам добавления в план проекта совещаний, ликвидации перегрузки ресурсов, оптимизации затрат и критического пути трудозатраты изменились для следующих групп:
% \begin{itemize}[label = ---]
% 	\item Нет назначения (сервер) --- уменьшились на 2\%;
% 	\item Анализ --- увеличились на 1\%;
% 	\item Мульти-медиа --- увеличились на 1\%.
% \end{itemize}

% Информация о затратах по группам ресурсов представлена для проекта на моменты его конечных состояний из ЛР №2 и ЛР №3 (рисунки~\ref{img:task2_3_group_cost}-\ref{img:task3_3_group_cost}).

% \imgw{task2_3_group_cost}{ht!}{0.7\textwidth}{Информация о затратах (ЛР №2)}
% \imgw{task3_3_group_cost}{ht!}{0.7\textwidth}{Информация о затратах (ЛР №3)}

% \newpage
% По результатам добавления в план проекта совещаний, ликвидации перегрузки ресурсов, оптимизации затрат и критического пути затраты изменились для следующих групп:
% \begin{itemize}[label = ---]
% 	\item Нет назначения (сервер) --- уменьшились на 1\%;
% 	\item Программирование --- уменьшились на 1\%;
% 	\item Анализ --- увеличились на 1\%;
% 	\item Документация --- увеличились на 1\%.
% \end{itemize}

% \newpage
% По результатам выполнения \textbf{ЛР №2} получились следующие соотношения \textbf{<<Затраты---Трудозатраты>>}:
% \begin{itemize}[label = ---]
% 	\item Анализ --- $\frac{10}{2}=5$;
% 	\item Программирование --- $\frac{50}{29}=1.72$;
% 	\item Нет назначения (сервер) --- $\frac{13}{32}=0.4$;
% 	\item Ввод данных --- $\frac{11}{26}=0.42$;
% 	\item Документация --- $\frac{2}{2}=1$.
% \end{itemize}

% По результатам выполнения \textbf{ЛР №3} получились следующие соотношения \textbf{<<Затраты---Трудозатраты>>}:
% \begin{itemize}[label = ---]
% 	\item Анализ --- $\frac{11}{3}=3.67$ (уменьшилось);
% 	\item Программирование --- $\frac{49}{29}=1.69$ (уменьшилось);
% 	\item Нет назначения (сервер) --- $\frac{12}{30}=0.4$;
% 	\item Ввод данных --- $\frac{11}{26}=0.42$;
% 	\item Документация --- $\frac{3}{2}=1.5$ (увеличилось).
% \end{itemize}

% Сохранение \textbf{базового плана} проекта приведено на рисунке \ref{img:task3_3_base_plan}. \imgw{task3_3_base_plan}{ht!}{0.5\textwidth}{Базовый план проекта}

% \subsection*{Вывод}
% Таким образом, соотношение \textbf{<<Затраты---Трудозатраты>>} уменьшилось для высокооплачиваемых специалистов (программистов) и самых высокооплачиваемых специалистов (аналитики), но увеличилось для низкооплачиваемого специалиста (технический писатель), что в результате привело к уменьшению стоимости проекта.

%\section{Вывод по работе}
%
%По результатам выполнения ЛР №2 программисты являются высокооплачиваемыми специалистами (на них приходится 50\% затрат и лишь 29\% трудозатрат проекта), значит, при сокращении времени их работы удастся сократить затраты проекта. В процессе выполнения ЛР №3 было выяснено, для оптимизации критического пути следует сократить время на задачи программирования. Т.к. от продолжительности этих задач зависит время начала следующих, сократив их продолжительность, удалось сократить общее время выполнения проекта. Это было реализовано посредством более равномерного перераспределения программистов на их задачи.
%
%После проведенных оптимизаций также удалось уменьшить соотношение \textbf{<<Затраты---Трудозатраты>>} для программистов (высокооплачиваемые) и аналитиков (самые высокооплачиваемые), при этом оно увеличилось для технического писателя (низкооплачиваемый). В результате удалось уменьшить затраты и продолжительность проекта.
%Дата окончания проекта сдвинулась с 25.09.23 на \textbf{20.07.23} (продолжительность всего проекта сократилась почти на 2 месяца и теперь не превышает установленного срока), а стоимость проекта уменьшилась с 49 849 до \textbf{48 480,18 рублей} (уменьшилась на 1 368,82 рублей и не превышает бюджета проекта).

