\chapter{Задание}

\section{Цель работы}

Продолжение знакомства с существующими методиками предварительной оценки параметров программного проекта и практическая оценка затрат по модели COCOMO II.

\section{Задание}

\begin{enumerate}
    \item Ознакомиться с прилагаемым к лабораторной работе теоретическим материалом (презентациями к лекции №8).
    \item На основе своего варианта задания рассчитать количество
функциональных точек для разрабатываемого программного
приложения. С этой целью разработать программный инструмент.
    \item Произвести оценку трудозатрат и длительности разработки по
методике COCOMO II с использованием моделей композиции
приложения и ранней разработки архитектуры.
    \item Определить среднюю численность команды разработчиков.
    \item На основе экспертной оценки стоимости человеко-месяца произвести предварительную оценку бюджета проекта.
    \item Дать заключение о применимости метода функциональных точек и модели COCOMO II, а также их сравнение с базовой моделью СОСОМО для решения поставленной задачи с учетом своего варианта.
\end{enumerate}

\chapter{Методика функциональных точек}

Данный метод используется для измерения производительности взамен устаревшего линейного подхода, где производительность измерялась количеством строк программного кода. 

Преимуществом данного метода является то, что поскольку применение функциональных точек основано на изучении требований, то оценка необходимых трудозатрат может быть выполнена на самых ранних стадиях работы над проектом.

Определение числа функциональных точек является методом количественной оценки ПО, применяемым для измерения функциональных характеристик процессов его разработки и сопровождения независимо от технологии, использованной для его реализации.

Трудоемкость вычисляется на основе функциональности разрабатываемой системы, которая, в свою очередь, определяется путем выявления функциональных типов —логических групп взаимосвязанных данных, используемых и поддерживаемых приложением, а также элементарных процессов, связанных с вводом и выводом информации.

\chapter{COCOMO II}

COCOMO II является развитием стандартного COCOMO. В методику входят три различные модели оценки стоимости.

\section{Модель композиции приложения}
Эта модель, которая подходит для проектов, созданных с помощью современных инструментальных средств. Единицей измерения служит объектная точка.

\section{Модель ранней разработки архитектуры}

Применяется для получения приблизительных оценок проектных затрат периода выполнения проекта перед тем как будет определена архитектура в целом. В этом случае используется небольшой набор новых драйверов затрат и новых уравнений оценки. В качестве единиц измерения используются функциональные точки либо KSLOC.

\section{Постархитектурная модель}

Наиболее детализированная модель \mbox{СОСОМО II}, которая используется после разработки архитектуры проекта. В состав этой модели включены новые драйверы затрат, новые правила подсчета строк кода, а также новые уравнения.

% TODO: вставить таблицу

