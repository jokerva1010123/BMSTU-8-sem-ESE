\chapter{Лабораторная работа}

\section{Описание проекта}

\section{Показатели проекта}

В данной работе рассматриваются модель композиции приложения и модель ранней разработки архитектуры. 

Показатели проекта:
\begin{itemize}[label = ---]
    \item новизна проекта (PREC) --- почти полное отсутствие прецедентов, в значительной мере непредсказуемый проект;
    \item гибкость процесса разработки (FLEX) --- точный, строгий процесс разработки;
    \item разрешение рисков в архитектуре системы (RESL) --- в целом (75\%)
    \item сплоченность команды (TEAM) --- взаимодействие как в едином целом;
    \item уровень развития процесса разработки (PMAT) --- уровень 2;
\end{itemize}

\section{Расчет по методу функциональных точек}

FTR --- количество связанных с каждым функциональным типом файлов типа ссылок.

DET --- количество связанных с каждым функциональным типом элементарных данных.

RET --- количество типов элементов записей.

EI (внешний ввод) --- элементарный процесс, перемещающий данные из внешней среды в приложение.

EO (внешний вывод) --- элементарный процесс, перемещающий данные, вычисленные в приложении, во внешнюю среду.

EQ (внешний запрос) --- элементарный процесс, состоящий из комбинации «запрос/ответ», не связанный с вычислением производных данных или обновлением внутренних логических файлов (базы данных).

ILF (внутренний логический файл) --- выделяемые пользователем логически связанные группы данных или блоки управляющей информации, которые поддерживаются внутри продукта и обслуживаются через внешние вводы.

EIF (внешний интерфейсный файл) --- выделяемые пользователем логически связанные группы данных или блоки управляющей информации, на которые ссылается продукт, но которые поддерживаются вне продукта.

В нашем приложении используются 4 внутренних файла: таблица с логинами и паролями, таблица с типом заявки, именем бумаги, ценой и количеством, таблица с названием бумаги. Также существует одна внешняя
таблица с информацией о бирже с названием бумаги, ценой и изменением.

\subsection*{Вычисление EI}
\begin{itemize}[label = ---]
    \item Регистрация
    
    FTR = 1 (один внутренний логический файл);
    
    DET = 4 (логин, пароль, номер водительского удостоверения и номер банковской карты).
    
    \item Оплата штрафа
    
    FTR = 1 (один внутренний логический файл);
    
    DET = 1 (дата оплаты).

    \item Добавление пользователя администратором;
    
    FTR = 1 (один внутренний логический файл);
    
    DET = 5 (логин, пароль, тип, номер карты, номер удостоверения).

    \item Редактирование пользователя администратором;
    
    FTR = 1 (один внутренний логический файл);
    
    DET = 5 (логин, пароль, тип, номер карты, номер удостоверения).
        
\end{itemize}

Уровень сложности – низкий.

\subsection*{Вычисление EO}

\begin{itemize}[label = ---]
    \item Вывод списка штрафов
    
    FTR = 1 (один внутренний логический файл);
    
    DET = 4 (номер, ФИО, дата, сумма).
    
    \item Вывод сообщения о статусе оплаты
    
    FTR = 1 (один внутренний логический файл);
    
    DET = 1 (статус оплаты).
    
    \item Вывод информации о пользователе
    
    FTR = 1 (один внутренний логический файл);
    
    DET = 6 (все поля из базы).

\end{itemize}

Уровень сложности – низкий.

\subsection*{Вычисление EQ}

\begin{itemize}[label = ---]
    \item Запрос на авторизацию
    
    FTR = 1 (один внутренний логический файл);
    
    DET = 3 (логин, пароль, флажок).
    
    \item Запрос на вывод списка
    
    FTR = 1 (один внутренний логический файл);
    
    DET = 4 (номер, ФИО, дата, сумма).
\end{itemize}

Уровень сложности – низкий.

\subsection*{Вычисление ILF}

\begin{itemize}[label = ---]
    \item ILF
    
    RET = 3 (элементы записи);
    
    DET = 6 (элементы данных).
\end{itemize}

Уровень сложности – низкий.

\subsection*{Вычисление EIF}

\begin{itemize}[label = ---]
    \item Веб
    
    RET = 3 (элементы записи);
    
    DET = 10 (элементы данных).
    
    \item Мобильное приложение
    
    RET = 3 (элементы записи);
    
    DET = 5 (элементы данных).
\end{itemize}

Уровень сложности --- низкий.

\section{Рузультат расчетов}

\subsection{Метод функциональных точек}

Нормированное количество функциональных точек: 47.94

Количество функциональных точек: 47

Количество строк исходного кода: 3590

\includeimage{task-1-1}{f}{h}{\textwidth}{ Расчет проекта по методу функциональных точек}

\clearpage

\subsection{Оценка по модели COCOMO II}

$p = 1.1964$

\includeimage{task-1-2}{f}{h}{\textwidth}{ Расчет проекта по методу композиции и ранней разработки архитектуры}

\subsection{Композиция приложения}

Формы:
\begin{itemize}[label = ---]
    \item страница регистрации --- простая (обращение к БД);
    \item страница авторизации --- простая (обращение к БД);
    \item страница штрафов --- средняя (обращение к БД ГИБДД);
    \item страница информации о пользователе --- простая (запрос к БД).
\end{itemize}

Отчеты:
\begin{itemize}[label = ---]
    \item отчет о пользователях --- простой (обращение к БД);
    \item отчет о штрафах --- средний (обращение к БД ГИБДД);
    \item отчет о платежах --- простой (обращение к БД).
\end{itemize}

Итого:
\begin{itemize}[label = ---]
    \item простые формы --- 3;
    \item средние формы --- 1;

    \item простые отчеты --- 2;
    \item средние отчеты --- 1;

    \item модули на ЯП третьего поколения --- 3;
    \item повторное использование --- 0 \%;
    \item опыт команды --- очень низкий;
\end{itemize}

\subsection{Модель ранней разработки архитектуры}

\begin{itemize}[label = ---]
\item PERS (возможности персонала) – очень высокий;
\item RCPX (надежность и уровень сложности разрабатываемой системы)
– очень высокий;
\item RUSE (повторное использование компонентов) – низкий;
\item PDIF (сложность платформы разработки) – номинальный;
\item PREX (опыт персонал) – низкий;
\item FCIL (средства поддержки) – очень высокий;
\item SCED (график работ) – низкий;
\item KSLOC = 3.6 (из метода функциональных точек).
\end{itemize}
