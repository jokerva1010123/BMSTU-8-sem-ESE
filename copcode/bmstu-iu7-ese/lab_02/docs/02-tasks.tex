\chapter{Лабораторная работа}

\section{Описание проекта}

Команда разработчиков из 16 человек занимается созданием карты города на основе собственного модуля отображения. Проект должен быть завершен в течение 6 месяцев. Бюджет проекта: 50 000 рублей.

\section{Создание списка ресурсов}

В соотвествии с заданием был заполнен лист ресурсов проекта.

\includeimage{task-1-0}{f}{h}{0.7\textwidth}{Настройка сведений о проекте}

\section{Назначение ресурсов задачам}

Задачам были назначены трудовые ресурсы при фиксированных трудозатратах.

\includeimage{task-2-0}{f}{h}{0.7\textwidth}{Ресурсный лист после назначения ресурсов}

В визуальном оптимизаторе можно увидеть наложения задач исполнителей.

\includeimage{task-2-1}{f}{h}{0.7\textwidth}{Наложение задач исполнителей}

По завершению внесения изменений получено следующее состояние проекта
\includeimage{task-2-2}{f}{h}{0.7\textwidth}{Состояние проекта после назначения ресурсов}

\section{Анализ затрат по группам ресурсов}

После группировки данных по группам ресурсов были получены диаграммы информации о затратах и трудозатратах.

\clearpage

\includeimage{task-3-0}{f}{h}{0.7\textwidth}{Информация о трудозатратах}

\includeimage{task-3-1}{f}{h}{0.7\textwidth}{Информация о затратах}

Коэффициент соотношения затрат к трудозатратам для отдельных групп:
\begin{itemize}
    \item программиронвание: $\frac{50}{29} = 1.7$;
    \item анализ: $\frac{10}{2} = 5$;
    \item ввод данных: $\frac{11}{25} = 0.4$;
    \item аренда сервера: $\frac{13}{32} = 0.4$;
\end{itemize}

Исходя из этого, видно, что наиболее высоко оплачиваемым отделом является \texttt{Анализ}. Так же, выскооплачиваемым является отдел \texttt{Программирование}.

Присутствует возможность оптимизации затрат за счет их сокращения на аренду сервера --- следует рассмотреть возможность частичной аренды и/или приобретения собственного оборудования.