\section*{Лабораторная работа}
\textbf{Содержание проекта:} Команда разработчиков из \textbf{16 человек} занимается созданием карты города на основе собственного модуля отображения. Проект должен быть завершен в течение \textbf{6 месяцев}. Бюджет проекта: 50 000 рублей.

\subsection*{Задание №1: Выравнивание загрузки ресурсов в проекте}

В задании №1 нужно было ликвидировать перегрузку ресурсов в проекте (рис. \ref{p1}).
\begin{figure}[!h]
	\centering
	\includegraphics[width=1\linewidth]{inc/img/1.png}
	\caption{Перегрузка ресурсов в проекте}
	\label{p1}
\end{figure}

По диаграмме Ганта видно, что некоторые ресурсы используются для нескольких задач одновременно, что решается переносом соответствующих задач на допустимую дату. На рисунке \ref{p2} изображена исправленная диаграмма Ганта.

\begin{figure}[!h]
	\centering
	\includegraphics[width=1\linewidth]{inc/img/2.png}
	\caption{Исправленная диаграмма Ганта}
	\label{p2}
\end{figure}

\newpage
\subsection*{Задание №2: Учет периодических задач в плане проекта}

При добавление еженедельных совещаний по средам, ресурсы снова стали перегруженными (рис. \ref{p3})

\begin{figure}[!h]
	\centering
	\includegraphics[width=1\linewidth]{inc/img/3.png}
	\caption{Перегрузка ресурсов после добавления совещаний}
	\label{p3}
\end{figure}

Большое количество перегрузок было устранено с помощью выравнивания (рис. \ref{p4})

\begin{figure}[!h]
	\centering
	\includegraphics[width=0.8\linewidth]{inc/img/4.png}
	\caption{Настройки выравнивания}
	\label{p4}
\end{figure}
\newpage
После избавления от перегрузок ресурсов, выполнения проекта увеличилось на 5 дней (рис. \ref{p5})

\begin{figure}[!h]
	\centering
	\includegraphics[width=1\linewidth]{inc/img/5.png}
	\caption{Диаграмма Ганта после выравнивания}
	\label{p5}
\end{figure}
\newpage
Кроме этого, несколько увеличился бюджет проекта (рис. \ref{p6}).

\begin{figure}[!h]
	\centering
	\includegraphics[width=1\linewidth]{inc/img/6.png}
	\caption{Затраты на совещания}
	\label{p6}
\end{figure}

\subsection*{Задание №3: Оптимизация критического пути}

Судя по критическому пути (рис. \ref{p7}), наибольшее влияние на срок реализации проекта оказывают задачи создания ядра GIS и создания интерфейса.

\begin{figure}[!h]
	\centering
	\includegraphics[width=1\linewidth]{inc/img/7.png}
	\caption{Критический путь}
	\label{p7}
\end{figure}

Для уменьшения срока реализации проекта были назначенны дополнительные ресурсы на задачи использующие программистов (рис. \ref{p8}).

\newpage
\begin{figure}[!h]
	\centering
	\includegraphics[width=1\linewidth]{inc/img/8.png}
	\caption{Дополнительное назначение ресурсов}
	\label{p8}
\end{figure}

На рисунке \ref{p9} можно увидеть диаграмму Ганта после ликвидации перегрузки ресурсов.

\begin{figure}[!h]
	\centering
	\includegraphics[width=1\linewidth]{inc/img/9.png}
	\caption{Диаграмма Ганта после устранения перегрузки}
	\label{p9}
\end{figure}

\newpage
Для уменьшения затрат было удвоено количество наборщиков данных, благодаря чему уменьшились затраты на сервер (рис. \ref{p10}).

\begin{figure}[!h]
	\centering
	\includegraphics[width=1\linewidth]{inc/img/10.png}
	\caption{Необходимый бюджет после снижения затрат}
	\label{p10}
\end{figure}

%\newpage
На рисунке \ref{d1} изображена диаграмма затрат по группам ресурсов.
\begin{figure}[!h]
	\centering
	\includegraphics[width=0.6\linewidth]{inc/img/d1.png}
	\caption{Диаграмма затрат по группам ресурсов}
	\label{d1}
\end{figure}

\newpage
На рисунке \ref{d2} изображена диаграмма трудозатрат по тем же группам ресурсов.
\begin{figure}[!h]
	\centering
	\includegraphics[width=0.6\linewidth]{inc/img/d2.png}
	\caption{Диаграмма трудозатрат по группам ресурсов}
	\label{d2}
\end{figure}

\section*{Выводы}
По сравнению с диаграммами из ЛР №2, соотношения затрат почти не изменились, если не считать уменьшение затрат на сервер, благодаря удвоению кол-ва наборщиков данных. По этой же причине трудозатраты сервера значительно уменьшились. В остальном соотношение трудозатрат к затратам почти не изменилось. Из этого можно сделать вывод что добавление еженедельных совещаний не значительно влияет на это соотношение.

По итогу проделанной работы были получены навыки оптимизации параметров проекта, выравнивания загрузки ресурсов и учета периодических задач, а также получилось минимизировать критический путь.